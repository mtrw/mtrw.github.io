% Options for packages loaded elsewhere
\PassOptionsToPackage{unicode}{hyperref}
\PassOptionsToPackage{hyphens}{url}
%
\documentclass[
]{article}
\usepackage{amsmath,amssymb}
\usepackage{iftex}
\ifPDFTeX
  \usepackage[T1]{fontenc}
  \usepackage[utf8]{inputenc}
  \usepackage{textcomp} % provide euro and other symbols
\else % if luatex or xetex
  \usepackage{unicode-math} % this also loads fontspec
  \defaultfontfeatures{Scale=MatchLowercase}
  \defaultfontfeatures[\rmfamily]{Ligatures=TeX,Scale=1}
\fi
\usepackage{lmodern}
\ifPDFTeX\else
  % xetex/luatex font selection
\fi
% Use upquote if available, for straight quotes in verbatim environments
\IfFileExists{upquote.sty}{\usepackage{upquote}}{}
\IfFileExists{microtype.sty}{% use microtype if available
  \usepackage[]{microtype}
  \UseMicrotypeSet[protrusion]{basicmath} % disable protrusion for tt fonts
}{}
\makeatletter
\@ifundefined{KOMAClassName}{% if non-KOMA class
  \IfFileExists{parskip.sty}{%
    \usepackage{parskip}
  }{% else
    \setlength{\parindent}{0pt}
    \setlength{\parskip}{6pt plus 2pt minus 1pt}}
}{% if KOMA class
  \KOMAoptions{parskip=half}}
\makeatother
\usepackage{xcolor}
\usepackage[margin=1in]{geometry}
\usepackage{graphicx}
\makeatletter
\def\maxwidth{\ifdim\Gin@nat@width>\linewidth\linewidth\else\Gin@nat@width\fi}
\def\maxheight{\ifdim\Gin@nat@height>\textheight\textheight\else\Gin@nat@height\fi}
\makeatother
% Scale images if necessary, so that they will not overflow the page
% margins by default, and it is still possible to overwrite the defaults
% using explicit options in \includegraphics[width, height, ...]{}
\setkeys{Gin}{width=\maxwidth,height=\maxheight,keepaspectratio}
% Set default figure placement to htbp
\makeatletter
\def\fps@figure{htbp}
\makeatother
\setlength{\emergencystretch}{3em} % prevent overfull lines
\providecommand{\tightlist}{%
  \setlength{\itemsep}{0pt}\setlength{\parskip}{0pt}}
\setcounter{secnumdepth}{-\maxdimen} % remove section numbering
\ifLuaTeX
  \usepackage{selnolig}  % disable illegal ligatures
\fi
\usepackage{bookmark}
\IfFileExists{xurl.sty}{\usepackage{xurl}}{} % add URL line breaks if available
\urlstyle{same}
\hypersetup{
  pdftitle={Getting started with R for statistics and bioinformatics},
  hidelinks,
  pdfcreator={LaTeX via pandoc}}

\title{Getting started with R for statistics and bioinformatics}
\author{}
\date{\vspace{-2.5em}}

\begin{document}
\maketitle

\begin{center}\rule{0.5\linewidth}{0.5pt}\end{center}

\section{Background}\label{background}

\subsubsection{\texorpdfstring{What is
\texttt{R}?}{What is R?}}\label{what-is-r}

R is a \emph{programming language}--a way for a human to tell a computer
to do things--designed specifically for analysing data, doing
statistics, and creating graphics. It is extremely popular, and
extremely useful. Users of R can create \emph{packages}, which add new
functions to R, and thanks to the existence of several thousand
publicly-available packages, R has an incredibly large range of
applications. In the field of bioinformatics, R's only main competitor
is the language Python, and knowing at least one of these is essential
to working as a bioinformatician.

\subsubsection{What makes this tutorial
different}\label{what-makes-this-tutorial-different}

There are plenty of R tutorials out there, and even some specific to
bioinformatics. Subtly, each reflects the preferences and styles of the
writer. You will eventually develop your own style, but your starting
point will greatly affect how fast you converge on a style that is
optimal for the work you're doing while reflecting your personality. And
yes, coding is a very expressive, creative activity and (in my opinion)
just as much a means of self-expression as painting or writing music.

This tutorial favours a style that: - Is \textbf{best for
bioinformatics}, where SPEED is hugely important. - Encourages
understanding of all the code at a deep level. This: - \textbf{Promotes
creativity}, by encouraging understanding of all the code at a deep
level, meaning you are free to express any idea that comes into your
head. `Black boxes' are avoided.

Because it aims to give you a deeper understanding, it spends a while on
the fundamentals. Much of what you learn in Part 1, you will not use in
everyday code because there are simpler ways to do it using more
sophisticated tools. BUT, as soon as you want to do something
\emph{special}, or \emph{creative}, or exceptionally \emph{efficient},
or \emph{new}, you will need to understand the fundamentals. Similarly,
knowing your fundamentals means you become far better at using more
advanced tools. Errors become less mysterious, results make more sense,
customising the tools becomes possible. This tutorial assumes you want
to truly learn the R language, not just use it to get one task done.

That said, the tutorial does not waste your time forcing you to learn
every single tiny feature and function in R. It focuses on examples to
help you get the key \textbf{concepts}. The best way to find and
understand specific features you want to use in your work is to find
code written by more experienced people who have done a task similar to
yours before, and learn from it---but once you've finished these
tutorials, the code you read from other people and from internet
searches will make much more sense and be much faster to absorb and
modify.

\end{document}
